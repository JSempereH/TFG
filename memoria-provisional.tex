\documentclass{article}
\usepackage{amsmath}
\usepackage{amssymb}

\newcommand{\R}{\mathbb{R}}

\title{Memoria provisional}

\begin{document}
\maketitle

%Introduccion: Revisión Bezier, motivación y objetivos
\section{Introducción}

\subsection{Curvas de Bézier}
Qué son, breve resumen de su historia y para qué son utilizadas.

\subsection{Motivación y objetivos}
Qué son las curvas fraccionarias generalizadas, ventajas, mencionar posibles usos. Objetivo: Aplicarlas al Data Fitting en Superficies de Riemann. Plan de trabajo.










%Curvas de Bezier generalizadas
\section{Curvas de Bézier fraccionarias generalizadas}

\subsection{Parámetros de forma}

Sea la curva de Bézier original definida por los $n+1$ puntos de control $P_i \in \R^m $:

\begin{equation*}
\alpha(t) = \sum_{i = 0}^{n}B_{i,n}(t)P_i
\end{equation*}
donde
\begin{equation*}
B_{i,n}(t) = \binom{n}{i}(1-t)^{n-i}t^i, \qquad  t\in[0,1]
\end{equation*}



Definimos la \textbf{Base de funciones Bernstein de grado n con n parámetros de forma} como:
\begin{gather*}
\hat{B}_{i,n}(t) = B_{i,n}(t)(1 + \frac{a_i}{n-i+1}(1-t) -\frac{a_{i+1}}{i+1}t), \qquad t\in[0,1] \\
a_0 = a_{n+1} = 0 \qquad -(n-i+1) < a_i < i, \qquad  i=0,1,...,n
\end{gather*}

donde $a_1,...,a_n$ son los \textit{parámetros de forma}. \newline

La base de funciones $\hat{B}_{i,n}(t)$ tiene las siguientes propiedades:
\begin{enumerate}
	\item $\hat{B}_{i,n}(t) \geq 0, \qquad t \in [0,1]$
	\item $\hat{B}_{i,n}(t) = \hat{B}_{n-i,n}(1-t)$, cuando $a_i = -a_{n-i+1}$
	\item $\hat{B}_{i,n}(t) = B_{i,n}(t)$ cuando $a_i = 0, \qquad i=1,...,n$.
	\item $\sum_{i=0}^{n}\hat{B}_{i,n}(t) = 1$
\end{enumerate}

\textbf{NOTA: ¿LO DEMUESTRO O MENCIONO EL ARTÍCULO DONDE SE DEMUESTRA? LA DEMOSTRACIÓN ES MECÁNICA}
\newline


%--------------------------------------------------------------
%		DEFINICIÓN CURVAS DE BÉZIER CON N PARÁMETROS DE FORMA
%--------------------------------------------------------------
Definimos la \textbf{Curva de Bézier de grado n con n parámetros de forma} como

\begin{equation}
\hat{\alpha}(t) = \sum_{i=0}^{n}\hat{B}_{i,n}(t)P_i, \qquad t \in [0,1]
\end{equation}

Por las propiedades de la base de funciones $\hat{B}_{i,n}$ se sigue que la nueva curva de Bézier cumple la propiedad del casco convexo.
\newline

Veamos el efecto que tiene en las curvas al variar los valores de los parámetros de forma.
El desarrollo de una curva de grado n con n parámetros de forma, con $\mathbf{P_0,P_1,...,P_n} \in \R^m$ puntos de control es:

\begin{gather*}
\hat{\alpha}(t) = B_{0,n}(t)(1 + 0 - a_1 t)\mathbf{P_0} + B_{1,n}(t)(1 + a_1 (1-t) - \frac{a_2}{2}t)\mathbf{P_1} + \\
+ B_{2,n}(t)(1 + \frac{a_2}{n-1}(1-t) - \frac{a_3}{3}t)\mathbf{P_2} + ... + \\
+ B_{i-1,n}(t)(1 + \frac{a_{i-1}}{n-i+2}(1-t) - \frac{a_i}{i}t) \mathbf{P_{i-1}} + \\
+ B_{i,n}(t)(1 + \frac{a_1}{n-i+1}(1-t) - \frac{a_{i+1}}{i+1}t) \mathbf{P_i} +...+ \\
+ B_{n,n}(t)(1 + a_n (1-t) + 0) \mathbf{P_n}
\end{gather*}














\end{document}